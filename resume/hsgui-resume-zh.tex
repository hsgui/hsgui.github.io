% available at http://www.latex-project.org/lppl/.

\documentclass[11pt,a4paper,sans,color]{moderncv}
% possible options include:
% font size:	('10pt', '11pt' and '12pt'),
% paper size:	('a4paper', 'letterpaper', 'a5paper', 'legalpaper', 'executivepaper' and 'landscape')
% font family:	('sans' and 'roman')

% moderncv theme
\moderncvstyle{classic}
% ‘casual’, ‘classic’, ‘oldstyle’ and ’banking’
\moderncvcolor{green}
% options: 'blue' (default), 'orange', 'green', 'red', 'purple' and 'grey'

%% encoding
\usepackage[noindent, UTF8]{ctex}

\usepackage[scale=0.85]{geometry}
\setlength{\hintscolumnwidth}{3cm}
%\setlength{\makecvtitlenamewidth}{10cm}
% for the 'classic' style, if you want to force the width allocated to your name and avoid line breaks. be careful though, the length is normally calculated to avoid any overlap with your personal info; use this at your own typographical risks...

\recomputelengths

% 个人信息
\firstname{}
\familyname{黄水桂}
\mobile{18551200993}
\email{hsguidemail@gmail.com}
\social[github]{hsgui}
\photo[64pt][0.4pt]{picture2.jpg}
%'64pt' is the height the picture must be resized to,
% 0.4pt is the thickness of the frame around it (put it to 0pt for no frame)
% 'picture' is the name of the picture file

\begin{document}

\maketitle

\section{工作背景}
\cventry{2014.4 -- 至今}{平台开发工程师}{Bing搜索后台(IndexServe)}{苏州}{}{
    \begin{itemize}
    \item 2015.4 -- 至今: 正排索引存储与在线服务
        \begin{itemize}
        \item 简介: 分布式数据存储系统,为整个Bing的网页和多媒体搜索的相关性排序提供数据服务
        \item 负责: 高可扩展性的正排索引格式,正排索引的存储方式以及在线服务。重点包括数据的正确性,完整性,高效率的数据写入(协同、增量、性能),控制在线写入时对在线服务的影响,以及错误处理
        \end{itemize}
    \item 2014.9 -- 2015.3: Bing搜索引擎的压力测试
        \begin{itemize}
        \item 负责: 对Bing进行压力测试,基于日志、性能指标等数据,监控系统的容量及抗压能力
        \end{itemize}
    \item 2014.4 -- 2014.9: Cortana语言模型的工具开发.
        \begin{itemize}
        \item 负责: 升级已有的模型训练、测试等工具,以及开发CUPhoneTester、CUPhoto Demo App、ActiveLearning Tool
        \end{itemize}
    \end{itemize}
}

\section{实习背景}
\cventry{2013.7 -- 2013.12}{软件开发工程师}{Bing中国}{北京}{}{
    \begin{itemize}
    \item Windows Phone版“必应词典”APP的电台、背单词功能的开发与测试
    \end{itemize}}
\cventry{2011.3 -- 2011.9}{软件测试开发工程师}{百度质量部}{北京}{}{
    \begin{itemize}
    \item “百度身边”的自动化测试开发,包括跟进产品功能开发进度,制定并实施测试计划;
    \end{itemize}}

\section{专业知识}
\cvlistitem{熟悉C、C++、C\#、Java}
\cvlistitem{熟悉数据结构、常用算法}
\cvlistitem{了解Python脚本语言; 了解Java Web开发以及前端开发技术; 了解Linux}
\cvlistitem{自学了机器学习、深度学习等课程;}

\section{教育背景}
\cventry{2011.9 -- 2014.4}{学术性硕士}{北京邮电大学网络技术研究院}{}{}{计算机科学与技术专业}
\cventry{2007.9 -- 2011.6}{工学学士}{北京邮电大学计算机科学与技术学院}{成绩前5\%,保送研究生}{}{计算机科学与技术专业}

\section{研究论文}
\cvlistitem{Huang, Shuigui, et al. "Polarity Identification of Sentiment Words Based on Emoticons." Computational Intelligence and Security (CIS), 2013 9th International Conference on. IEEE, 2013.}

\section{外语技能}
\cvitemwithcomment{英语}{CET-6 -- 527, CET-4 -- 526}{具有良好的英文阅读能力}

\section{在校所获奖励}
\cvitem{2012}{北京邮电大学 \textbf{优秀研究生干部}}
\cvitem{2009}{全国大学生数学建模竞赛 \textbf{北京市三等奖}}
\cvitem{2008}{北京市大学生数学竞赛 \textbf{甲组三等奖}}
\cvitem{2011,2012}{北京邮电大学 \textbf{研究生一等奖学金}}
\cvitem{2008,2009}{\textbf{国家励志奖学金}}

\end{document}
