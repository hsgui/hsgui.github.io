% available at http://www.latex-project.org/lppl/.

\documentclass[11pt,a4paper,sans,color]{moderncv}
% possible options include:
% font size:	('10pt', '11pt' and '12pt'),
% paper size:	('a4paper', 'letterpaper', 'a5paper', 'legalpaper', 'executivepaper' and 'landscape')
% font family:	('sans' and 'roman')

% moderncv theme
\moderncvstyle{classic}
% ‘casual’, ‘classic’, ‘oldstyle’ and ’banking’
\moderncvcolor{green}
% options: 'blue' (default), 'orange', 'green', 'red', 'purple' and 'grey'

%% encoding
\usepackage[noindent, UTF8]{ctex}

\usepackage[scale=0.85]{geometry}
\setlength{\hintscolumnwidth}{3cm}
%\setlength{\makecvtitlenamewidth}{10cm}
% for the 'classic' style, if you want to force the width allocated to your name and avoid line breaks. be careful though, the length is normally calculated to avoid any overlap with your personal info; use this at your own typographical risks...

\recomputelengths

% 个人信息
\firstname{}
\familyname{黄水桂}
\mobile{18551200993}
\email{hsguidemail@gmail.com}
\social[github]{hsgui}
\photo[64pt][0.4pt]{picture2.jpg}
%'64pt' is the height the picture must be resized to,
% 0.4pt is the thickness of the frame around it (put it to 0pt for no frame)
% 'picture' is the name of the picture file

\begin{document}

\maketitle

\section{教育背景}
\cventry{2011.9 -- 2014.4}{学术性硕士}{北京邮电大学网络技术研究院}{}{}{计算机科学与技术专业}
\cventry{2007.9 -- 2011.6}{工学学士}{北京邮电大学计算机科学与技术学院}{成绩前5\%}{}{计算机科学与技术专业}

\section{工作背景}
\cventry{2014.4 -- 至今}{软件开发工程师}{微软中国ASG}{苏州}{}{
	\begin{itemize}
	\item 2014.4 -- 2014.9: Conversation Understanding Toolkit Team.
		\begin{itemize}
		\item 简介: 开发语言模型的训练、测试及模型修正等工具,便于语言模型的生成、调试及优化
		\item 负责: CUTrain、CUTester、CUHotfixTool的升级
        \item 负责: CUPhoneTester、CUPhoto App(Ignite Demo App)、ActiveLearning Tool开发
		\end{itemize}
	\item 2014.9 -- 2015.3: Index Serve Capacity Engineering Team.
		\begin{itemize}
		\item 简介: 周期性对Bing进行压力测试,日志等数据的采集,以便监控系统的容量及抗压能力
		\item 负责: 基于系统日志,分析Bing stack各层的时延及负载容量
        \item 负责: 将整个压力测试重构到分布式的自动调度平台上
		\end{itemize}
	\item 2015.3 -- 至今: Index Serve PerDocIndex Team.
		\begin{itemize}
		\item 简介: Web Page的正排索引,该索引中包含文档的各种属性,用来提升搜索速度和相关度,对性能非常敏感
		\item 负责: 增量式更新正排索引,独立正排索引服务,
        \item 负责: 升级索引文件格式,以及将PDI升级为KV存储(支持point update)
		\end{itemize}
	\end{itemize}
}

\section{研究生项目背景}
\cventry{2011.9 -- 2012.6}{移动社会搜索的问答系统}{Nokia合作项目}{}{}{
    \begin{itemize}
    \item 负责:系统后台的搭建,以及后台接口的设计与实现.
    \item 相关技术:JAVA、Spring、Mybatis、MySql、Maven
    \end{itemize}}
\cventry{2012.9 -- 2013.12}{社会化媒体的情感倾向性分析系统}{自选项目}{}{}{
    \begin{itemize}
    \item 负责:系统的整体设计与实现,包括系统的后台算法实现以及前端页面的展示;
    \item Huang, Shuigui, et al. \textbf{"Polarity Identification of Sentiment Words Based on Emoticons."} Computational Intelligence and Security (CIS), 2013 9th International Conference on. IEEE, 2013.
    \item 相关技术:前端Bootstrap、HighCharts、jQuery等,后台Java、Spring、Maven等;
    \end{itemize}}

\section{实习背景}
\cventry{2011.3 -- 2011.9}{软件测试开发工程师}{百度质量部}{北京}{}{
	\begin{itemize}
	\item “百度身边”的自动化测试开发,包括跟进产品功能开发进度,制定并实施测试计划;
	\end{itemize}}
	\cventry{2013.7 -- 2013.12}{软件开发工程师}{微软中国STCA}{北京}{}{
	\begin{itemize}
	\item Windows Phone版“必应词典”APP的电台、背单词功能的开发与测试
	\end{itemize}}

\section{专业知识}
\cvlistitem{熟悉C、C++、C\#}
\cvlistitem{熟悉数据结构、常用算法}
\cvlistitem{了解Python、 Java语言; 了解Java Web开发以及前端开发技术; 了解Linux}
\cvlistitem{自学了机器学习、深度学习等课程;}

\section{外语技能}
\cvitemwithcomment{英语}{CET-6 -- 527, CET-4 -- 526}{具有良好的英文阅读能力}

\section{所获奖励}
\cvitem{2012}{北京邮电大学 \textbf{优秀研究生干部}}
\cvitem{2009}{全国大学生数学建模竞赛 \textbf{北京市三等奖}}
\cvitem{2008}{北京市大学生数学竞赛 \textbf{甲组三等奖}}
\cvitem{2011,2012}{北京邮电大学 \textbf{研究生一等奖学金}}
\cvitem{2008,2009}{\textbf{国家励志奖学金}}

\section{学生工作}
\cvlistitem{担任研究中心班主任}

\end{document}

